\documentclass[14pt]{article}
\usepackage{amsmath}
\usepackage{listings} % For writing code see http://ctan.org/pkg/listings
\usepackage{graphicx}
\usepackage{float}
\usepackage[margin=1.0in]{geometry}
\usepackage{hyperref}

\title{Stellar tagging method}
\author{Nicol\'as Garavito-Camargo}


\begin{document}
\maketitle

All based on Chervin's work! Laporte+13, +15\\

\section{Method:}


Repository with code and examples available
\href{https://github.com/jngaravitoc/tracers_triaxial/tree/interpolation}{here}.

\begin{equation}
  w(E) = \dfrac{N_*(E)}{N(E)} = \dfrac{f_*(E)g(E)}{N(E)}
\end{equation}

\begin{itemize}
  \item $N(E)dE = g(E)f(E)dE$: Differential energy distribution: fraction of stars
    that have energies in the range $E+dE$.
  \item $g(E)$: Density of states: The volume of phase space per unit energy.
  \item $f(E)$: Distribution function: Probability of finding a particle with
    energy $E+dE$ in the phase space. 
\end{itemize}

\subsection{Differential energy distribution:}

Number of stars in each energy bin (Histogram of the energies).

\begin{figure}[H]
  \centering
  \includegraphics[scale=0.5]{../N_E.pdf}
  \caption{Differential energy distribution  $N(E)$ for a Hernquist DM halo (solid black
  line) and a smoothed $N(E)$}
\end{figure}


\subsection{Distribution function:}


Assuming:

\begin{itemize}
  \item Virial equilibrium.
  \item Spherical symmetry.  
\end{itemize}

Eddington's Formula: 

\begin{equation}
  f_*(\mathcal{E}) = \dfrac{1}{\sqrt{8}\pi^2}\int_0^{\mathcal{E}} \dfrac{d
  \nu}{\sqrt{\mathcal{E}- \Psi}}  \dfrac{d^2 \nu}{d \Psi^2} +
  \dfrac{1}{\sqrt{\mathcal{E}}}\dfrac{d \nu}{d\Psi} \Bigg\vert _{\Psi=0} 
\end{equation}

Where $\nu=\rho/M$ is the number density of tracers, $\mathcal{E}=-E$,
$\Psi=-\Phi$.

Computing $\dfrac{d \nu}{d \psi}$ and $\dfrac{d^2 \nu}{d \psi^2}$


\begin{figure}[H]
  \centering
  \includegraphics[scale=0.5]{derivatives_dnu_dpsi.pdf}
  \includegraphics[scale=0.5]{DF_Hernquist.pdf}
  \caption{\textit{Right panel:} derivatives of the tracers density with respect
  to the potential $\Psi$. \textit{Left panel}: DF of a Hernquist halo.}
\end{figure}



\subsection{Density of states:}


The density of states is computed as:

\begin{equation}
  g(E) = (4\pi)^2 \int_0^{r_E} r^2 \sqrt{2(E - \phi(r))} dr
\end{equation}


\begin{figure}[H]
  \centering
  \includegraphics[scale=0.5]{gE.pdf}
  \caption{Density of States for the Hernquist Halo.}
\end{figure}

\subsection{Assigning weights:}

The weights $w(E)$ are computed for each energy bin. Then all the DM particles
are assigned a weight corresponding to the particle energy.

\subsection{Test I: Building a stellar halo from a DM Hernquist halo.}

\begin{equation}
  m_{*,i} = w_i m_{DM} 
\end{equation}

\begin{figure}[H]
   \includegraphics[scale=0.5]{Hernquist_stellar_halos.pdf}
   \includegraphics[scale=0.5]{NFW_stellar_halos.pdf}
   \caption{Density profiles of the stellar built from a Hernquist DM halo.
   \textit{Right panel} show a Hernquist stellar halo for different scale
   lengths. \textit{Left panel} show an NFW stellar halo for different
   concentrations.}
\end{figure}

\subsection{Test II: Building the velocity dispersion profile. }

\begin{equation}
  v_{*,i} = w_i m_{DM} 
\end{equation}


\begin{equation}
  \sigma_r = \dfrac{\sum_i w_i (v_{r,i}-\bar{v})^2}{\sum_i w_i}
\end{equation}

\begin{figure}[H]
  \centering
  \includegraphics[scale=0.5]{velocity_dispersion_tracers_hernquist.pdf}
  \caption{Velocity dispersion profile for a Hernquist DM halo (solid black line)  and
  the corresponding stellar halo and two different times (cyan and purple line).
  The velocity dispersion profile of the stellar particles decrease faster with
  radius that the velocity dispersion of the DM halo.}
\end{figure}



\end{document}

